\documentclass{article}
\usepackage{fancyhdr}
\usepackage[utf8]{inputenc}
\usepackage{mathtools}
\usepackage{geometry}
\usepackage{amssymb}
\usepackage{mathrsfs}
\usepackage{scrextend}

\title{\LARGE{\textbf{Métodos numéricos: Interpolación.}}}
\author{1º DGIIM}
\date{}

\pagestyle{fancy}
\fancyhf{}
\fancyhead[LO,LE]{\textbf{Tema 4. Interpolación.}}
\fancyhead[RO,RE]{1º DGIIM}
\fancyfoot[CO,CE]{\thepage}

\begin{document}

\maketitle

\hrulefill

%					%
%	INTRODUCCIÓN	%
%					%
\section{Introducción}

Sea $p(x)\in\mathbb{R}[x]$ y $a,\alpha\in\mathbb{R}$, entonces ese polinomio se puede escribir de la siguiente manera:
$$p(x)=(x-a)q(x)+\alpha$$
Donde $(x-a)$ es el divisor, $q(x)$ el cociente y $\alpha$ el resto
\end{document}